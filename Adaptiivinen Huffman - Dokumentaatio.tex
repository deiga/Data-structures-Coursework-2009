\documentclass[11pt,a4paper,draft]{article}

\usepackage[utf8]{inputenc}
\usepackage[finnish]{babel}
\usepackage[T1]{fontenc}

\usepackage{amsmath}
\usepackage{amssymb}
\usepackage{amsfonts}
\usepackage{amsthm}

\newcommand{\HRule}{\rule{\linewidth}{0.5mm}}

\usepackage{geometry}
\geometry{a4paper,left=30mm,right=30mm,top=45mm,bottom=25mm}

\author{Timo Sand}
\title{Adaptiivnen Huffman - Dokumentaatio}

\begin{document}
	
	
	\pagestyle{headings}
	\begin{titlepage}

	
	\begin{center}
		\textsc{\LARGE Helsingin Yliopisto}\\[1.5cm]
		\textsc{\LARGE Tietojenkäsittelytieteen laitos}\\[1.0cm]
		\textsc{\Large Tietorakenteiden harjoitustyö}\\[0.5cm]
		
		\HRule \\[0.4cm]
		{ \huge \bfseries Adaptive Huffman encoding}\\[0.4cm]

		\HRule \\[1.5cm]
		
		Timo \textsc{Sand}\\
			timo.sand@cs.helsinki.fi

		
		\vfill
		
		{\large \today}
		
	\end{center}
	
\end{titlepage}
	\tableofcontents
	\thispagestyle{empty}
	\newpage
	\setcounter{page}{1}
	
	\section{Aiheen määrittely} % (fold)
	\label{sec:aiheen_maarittely}
	
	Harjoitustyön aiheena on tehdä ohjelma, jokaa pystyy pakkaamaan ja purkamaan 
	esim. tekstitiedoston. Tähän tarkoitukseen käytetään “Adaptiivista Huffman” -koodausta. 
	Huffman -koodauksessa pakattavaa dataa käsitellään yksi merkki kerrallaan, eli luetaan 
	merkit bittijonona. Kukin merkki koodataan bittijonoksi, joka on sitä lyhyempi, mitä 
	useammin merkki esiintyy datassa. Adaptiivisessa Huffman-koodauksessa luodaan 
	binääripuu ns. “lennosta”, eli jokainen luettu merkki päivittää puuta ja täten merkkejen 
	bittijonoja. Purku tapahtuu samassa järjestyksessä, eli luetaan bittijono ja käännetään se merkiksi, joka lisätään puuhun. 
	Kun uusi merkki luetaan ensimmäistä kertaa puuhun, niin sen desimaaliesitys kirjoitetaan kokonaisena bittijonona 
	tiedostoon, jotta siitä pystytään rakentamaan puuta.
	\\\\
	Käytettyjen työtuntien määrä: n. 80-85
	
	% section aiheen_määrittely (end)
	
	\section{Käyttöohje} % (fold)
	\label{sec:kayttoohje}
	
	Ohjelma käynnistetään komennolla huffman.sh. Ilman parametrejä se tulostaa ohjeet käyttäjälle. 
	Parametrillä \emph{-d} ohjelma ajetaan gdb:ssä ja tiedostopolku paramaterinä käynnistää itse ohjelman.
	
	% section käyttöohje (end)
	
	\section{Algoritmit ja tietorakenteet} % (fold)
	\label{sec:algoritmit_ja_tietorakenteet}
	
		\subsection{Huffman-puu} % (fold)
		\label{sub:huffman_puu}
		
			Huffman-puu on binääripuu. Huffman-puussa on solmuja \( 2n-1 \), joten, kun oletetaan 
			että käytetään ainoastaan ASCII merkkejä, niin voidaan sanoa, kun lehtiä on 257, 
			että solmuja on maksimissaan 513. Huffmann-puu on aina täysi, 
			elikkä jokaisella solmulla on 0 tai 2 lasta. Lehtien määrä siis merkkien desimaaliesityksen 
			mukaan 0-255, sekä NYA-solmu, joka siis kuvastaa ei vielä koodattuja merkkejä.
			
			Binääripuun lisäksi käytetään kahta taulukkoa, toisessa on suorat viitteet puun alkioihin, täten kaikki 
			puuhun liittyvät operaatiot tulevat vakioaikaiseksi ja toisessa taulukossa on kaikki puun solmujen painot. 
			Ensimmäinen indeksoidaan alkioitten merkin bittiesityksen mukaan ja toinen solmuje indeksin mukaan. 
			
		
		% subsection huffman_puu (end)
		\subsection{FGK-algoritmi} % (fold)
		\label{sub:fgk_algoritmi}
			
			Algoritmi FGK on kolmen miehen kehittämä algoritmi: Faller, Gallager ja Knuth. Algoritmin pohjana on solmujen sisar-ehto.
			Binääripuun solmuilla, jossa ei ole negatiivisiä painoarvoja, on sisar-ehto voimassa, jos jokaisella solmulla on sisar 
			ja jos solmut voidaan numeroida nousevassa järjestyksessä, jossa jokainen solmu on sen sisar-solmun vieressä. 
			Myöskin solmun vanhempi on koreammalla numerojärjestyksessä.
			
			Algoritmin toimintaperiaate on seuraavanlainen. Niin pakkaus kun purku rakentaa syötteenä annetusta tiedostosta huffman-puun samalla tavalla.
			Jokaista merkkiä kohden lähetetään koodijono puuhun, joka sitten päivitetään.
		
		% subsection fgk_algoritmi (end)
	
	% section algoritmit_ja_tietorakenteet (end)
	
	\section{Testaus} % (fold)
	\label{sec:testaus}
	
		Ohjelman pakkaamista on testattu syötteillä: lorem.txt ja ly-ebook.txt
		Ohjelma purkua on testattu syötteilä: lorem.bin ja ly-ebook.bin
		Kummatkin testisyötteet ovat mukana. Kaikilla testeillä tähän asti ohjelma suoriutuu. 
		Ohjelmaa on myös testattu intensiivisesti toteutuksen aikana.
	
	% section testaus (end)
	
	\section{Toteutuksen puutteet} % (fold)
	\label{sec:toteutuksen_puutteet}
	
	Toteutus ei tällä hetkellä osaa kirjoittaa bittijonoja, joten pakkaamisen sijaan, se tekee alkuperäisistä tiedostoista isompia.
	Purun yhteydessä ei osata käsitellä tiedoston lopussa olevia roskabittejä, joten tiedoston loppuun saatetaan kirjoittaa "roskaa".
	
	% section toteutuksen_puutteet (end)
	
	\begin{thebibliography}{99}
		\bibitem{applet} \emph{http://www.cs.sfu.ca/CC/365/li/squeeze/AdaptiveHuff.html} - Java appletti huffmannin toiminnan visualisoimiseksi
		\bibitem{duke} \emph{http://www.cs.duke.edu/csed/curious/compression/adaptivehuff.html} - Sivusto, jossa hyvin selitetty ja havainnollistavat kaaviot.
	\end{thebibliography}
	
\end{document}